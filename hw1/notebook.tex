
% Default to the notebook output style

    


% Inherit from the specified cell style.




    
\documentclass[11pt]{article}

    
    
    \usepackage[T1]{fontenc}
    % Nicer default font (+ math font) than Computer Modern for most use cases
    \usepackage{mathpazo}

    % Basic figure setup, for now with no caption control since it's done
    % automatically by Pandoc (which extracts ![](path) syntax from Markdown).
    \usepackage{graphicx}
    % We will generate all images so they have a width \maxwidth. This means
    % that they will get their normal width if they fit onto the page, but
    % are scaled down if they would overflow the margins.
    \makeatletter
    \def\maxwidth{\ifdim\Gin@nat@width>\linewidth\linewidth
    \else\Gin@nat@width\fi}
    \makeatother
    \let\Oldincludegraphics\includegraphics
    % Set max figure width to be 80% of text width, for now hardcoded.
    \renewcommand{\includegraphics}[1]{\Oldincludegraphics[width=.8\maxwidth]{#1}}
    % Ensure that by default, figures have no caption (until we provide a
    % proper Figure object with a Caption API and a way to capture that
    % in the conversion process - todo).
    \usepackage{caption}
    \DeclareCaptionLabelFormat{nolabel}{}
    \captionsetup{labelformat=nolabel}

    \usepackage{adjustbox} % Used to constrain images to a maximum size 
    \usepackage{xcolor} % Allow colors to be defined
    \usepackage{enumerate} % Needed for markdown enumerations to work
    \usepackage{geometry} % Used to adjust the document margins
    \usepackage{amsmath} % Equations
    \usepackage{amssymb} % Equations
    \usepackage{textcomp} % defines textquotesingle
    % Hack from http://tex.stackexchange.com/a/47451/13684:
    \AtBeginDocument{%
        \def\PYZsq{\textquotesingle}% Upright quotes in Pygmentized code
    }
    \usepackage{upquote} % Upright quotes for verbatim code
    \usepackage{eurosym} % defines \euro
    \usepackage[mathletters]{ucs} % Extended unicode (utf-8) support
    \usepackage[utf8x]{inputenc} % Allow utf-8 characters in the tex document
    \usepackage{fancyvrb} % verbatim replacement that allows latex
    \usepackage{grffile} % extends the file name processing of package graphics 
                         % to support a larger range 
    % The hyperref package gives us a pdf with properly built
    % internal navigation ('pdf bookmarks' for the table of contents,
    % internal cross-reference links, web links for URLs, etc.)
    \usepackage{hyperref}
    \usepackage{longtable} % longtable support required by pandoc >1.10
    \usepackage{booktabs}  % table support for pandoc > 1.12.2
    \usepackage[inline]{enumitem} % IRkernel/repr support (it uses the enumerate* environment)
    \usepackage[normalem]{ulem} % ulem is needed to support strikethroughs (\sout)
                                % normalem makes italics be italics, not underlines
    

    
    
    % Colors for the hyperref package
    \definecolor{urlcolor}{rgb}{0,.145,.698}
    \definecolor{linkcolor}{rgb}{.71,0.21,0.01}
    \definecolor{citecolor}{rgb}{.12,.54,.11}

    % ANSI colors
    \definecolor{ansi-black}{HTML}{3E424D}
    \definecolor{ansi-black-intense}{HTML}{282C36}
    \definecolor{ansi-red}{HTML}{E75C58}
    \definecolor{ansi-red-intense}{HTML}{B22B31}
    \definecolor{ansi-green}{HTML}{00A250}
    \definecolor{ansi-green-intense}{HTML}{007427}
    \definecolor{ansi-yellow}{HTML}{DDB62B}
    \definecolor{ansi-yellow-intense}{HTML}{B27D12}
    \definecolor{ansi-blue}{HTML}{208FFB}
    \definecolor{ansi-blue-intense}{HTML}{0065CA}
    \definecolor{ansi-magenta}{HTML}{D160C4}
    \definecolor{ansi-magenta-intense}{HTML}{A03196}
    \definecolor{ansi-cyan}{HTML}{60C6C8}
    \definecolor{ansi-cyan-intense}{HTML}{258F8F}
    \definecolor{ansi-white}{HTML}{C5C1B4}
    \definecolor{ansi-white-intense}{HTML}{A1A6B2}

    % commands and environments needed by pandoc snippets
    % extracted from the output of `pandoc -s`
    \providecommand{\tightlist}{%
      \setlength{\itemsep}{0pt}\setlength{\parskip}{0pt}}
    \DefineVerbatimEnvironment{Highlighting}{Verbatim}{commandchars=\\\{\}}
    % Add ',fontsize=\small' for more characters per line
    \newenvironment{Shaded}{}{}
    \newcommand{\KeywordTok}[1]{\textcolor[rgb]{0.00,0.44,0.13}{\textbf{{#1}}}}
    \newcommand{\DataTypeTok}[1]{\textcolor[rgb]{0.56,0.13,0.00}{{#1}}}
    \newcommand{\DecValTok}[1]{\textcolor[rgb]{0.25,0.63,0.44}{{#1}}}
    \newcommand{\BaseNTok}[1]{\textcolor[rgb]{0.25,0.63,0.44}{{#1}}}
    \newcommand{\FloatTok}[1]{\textcolor[rgb]{0.25,0.63,0.44}{{#1}}}
    \newcommand{\CharTok}[1]{\textcolor[rgb]{0.25,0.44,0.63}{{#1}}}
    \newcommand{\StringTok}[1]{\textcolor[rgb]{0.25,0.44,0.63}{{#1}}}
    \newcommand{\CommentTok}[1]{\textcolor[rgb]{0.38,0.63,0.69}{\textit{{#1}}}}
    \newcommand{\OtherTok}[1]{\textcolor[rgb]{0.00,0.44,0.13}{{#1}}}
    \newcommand{\AlertTok}[1]{\textcolor[rgb]{1.00,0.00,0.00}{\textbf{{#1}}}}
    \newcommand{\FunctionTok}[1]{\textcolor[rgb]{0.02,0.16,0.49}{{#1}}}
    \newcommand{\RegionMarkerTok}[1]{{#1}}
    \newcommand{\ErrorTok}[1]{\textcolor[rgb]{1.00,0.00,0.00}{\textbf{{#1}}}}
    \newcommand{\NormalTok}[1]{{#1}}
    
    % Additional commands for more recent versions of Pandoc
    \newcommand{\ConstantTok}[1]{\textcolor[rgb]{0.53,0.00,0.00}{{#1}}}
    \newcommand{\SpecialCharTok}[1]{\textcolor[rgb]{0.25,0.44,0.63}{{#1}}}
    \newcommand{\VerbatimStringTok}[1]{\textcolor[rgb]{0.25,0.44,0.63}{{#1}}}
    \newcommand{\SpecialStringTok}[1]{\textcolor[rgb]{0.73,0.40,0.53}{{#1}}}
    \newcommand{\ImportTok}[1]{{#1}}
    \newcommand{\DocumentationTok}[1]{\textcolor[rgb]{0.73,0.13,0.13}{\textit{{#1}}}}
    \newcommand{\AnnotationTok}[1]{\textcolor[rgb]{0.38,0.63,0.69}{\textbf{\textit{{#1}}}}}
    \newcommand{\CommentVarTok}[1]{\textcolor[rgb]{0.38,0.63,0.69}{\textbf{\textit{{#1}}}}}
    \newcommand{\VariableTok}[1]{\textcolor[rgb]{0.10,0.09,0.49}{{#1}}}
    \newcommand{\ControlFlowTok}[1]{\textcolor[rgb]{0.00,0.44,0.13}{\textbf{{#1}}}}
    \newcommand{\OperatorTok}[1]{\textcolor[rgb]{0.40,0.40,0.40}{{#1}}}
    \newcommand{\BuiltInTok}[1]{{#1}}
    \newcommand{\ExtensionTok}[1]{{#1}}
    \newcommand{\PreprocessorTok}[1]{\textcolor[rgb]{0.74,0.48,0.00}{{#1}}}
    \newcommand{\AttributeTok}[1]{\textcolor[rgb]{0.49,0.56,0.16}{{#1}}}
    \newcommand{\InformationTok}[1]{\textcolor[rgb]{0.38,0.63,0.69}{\textbf{\textit{{#1}}}}}
    \newcommand{\WarningTok}[1]{\textcolor[rgb]{0.38,0.63,0.69}{\textbf{\textit{{#1}}}}}
    
    
    % Define a nice break command that doesn't care if a line doesn't already
    % exist.
    \def\br{\hspace*{\fill} \\* }
    % Math Jax compatability definitions
    \def\gt{>}
    \def\lt{<}
    % Document parameters
    \title{Assignment\_01}
    
    
    

    % Pygments definitions
    
\makeatletter
\def\PY@reset{\let\PY@it=\relax \let\PY@bf=\relax%
    \let\PY@ul=\relax \let\PY@tc=\relax%
    \let\PY@bc=\relax \let\PY@ff=\relax}
\def\PY@tok#1{\csname PY@tok@#1\endcsname}
\def\PY@toks#1+{\ifx\relax#1\empty\else%
    \PY@tok{#1}\expandafter\PY@toks\fi}
\def\PY@do#1{\PY@bc{\PY@tc{\PY@ul{%
    \PY@it{\PY@bf{\PY@ff{#1}}}}}}}
\def\PY#1#2{\PY@reset\PY@toks#1+\relax+\PY@do{#2}}

\expandafter\def\csname PY@tok@w\endcsname{\def\PY@tc##1{\textcolor[rgb]{0.73,0.73,0.73}{##1}}}
\expandafter\def\csname PY@tok@c\endcsname{\let\PY@it=\textit\def\PY@tc##1{\textcolor[rgb]{0.25,0.50,0.50}{##1}}}
\expandafter\def\csname PY@tok@cp\endcsname{\def\PY@tc##1{\textcolor[rgb]{0.74,0.48,0.00}{##1}}}
\expandafter\def\csname PY@tok@k\endcsname{\let\PY@bf=\textbf\def\PY@tc##1{\textcolor[rgb]{0.00,0.50,0.00}{##1}}}
\expandafter\def\csname PY@tok@kp\endcsname{\def\PY@tc##1{\textcolor[rgb]{0.00,0.50,0.00}{##1}}}
\expandafter\def\csname PY@tok@kt\endcsname{\def\PY@tc##1{\textcolor[rgb]{0.69,0.00,0.25}{##1}}}
\expandafter\def\csname PY@tok@o\endcsname{\def\PY@tc##1{\textcolor[rgb]{0.40,0.40,0.40}{##1}}}
\expandafter\def\csname PY@tok@ow\endcsname{\let\PY@bf=\textbf\def\PY@tc##1{\textcolor[rgb]{0.67,0.13,1.00}{##1}}}
\expandafter\def\csname PY@tok@nb\endcsname{\def\PY@tc##1{\textcolor[rgb]{0.00,0.50,0.00}{##1}}}
\expandafter\def\csname PY@tok@nf\endcsname{\def\PY@tc##1{\textcolor[rgb]{0.00,0.00,1.00}{##1}}}
\expandafter\def\csname PY@tok@nc\endcsname{\let\PY@bf=\textbf\def\PY@tc##1{\textcolor[rgb]{0.00,0.00,1.00}{##1}}}
\expandafter\def\csname PY@tok@nn\endcsname{\let\PY@bf=\textbf\def\PY@tc##1{\textcolor[rgb]{0.00,0.00,1.00}{##1}}}
\expandafter\def\csname PY@tok@ne\endcsname{\let\PY@bf=\textbf\def\PY@tc##1{\textcolor[rgb]{0.82,0.25,0.23}{##1}}}
\expandafter\def\csname PY@tok@nv\endcsname{\def\PY@tc##1{\textcolor[rgb]{0.10,0.09,0.49}{##1}}}
\expandafter\def\csname PY@tok@no\endcsname{\def\PY@tc##1{\textcolor[rgb]{0.53,0.00,0.00}{##1}}}
\expandafter\def\csname PY@tok@nl\endcsname{\def\PY@tc##1{\textcolor[rgb]{0.63,0.63,0.00}{##1}}}
\expandafter\def\csname PY@tok@ni\endcsname{\let\PY@bf=\textbf\def\PY@tc##1{\textcolor[rgb]{0.60,0.60,0.60}{##1}}}
\expandafter\def\csname PY@tok@na\endcsname{\def\PY@tc##1{\textcolor[rgb]{0.49,0.56,0.16}{##1}}}
\expandafter\def\csname PY@tok@nt\endcsname{\let\PY@bf=\textbf\def\PY@tc##1{\textcolor[rgb]{0.00,0.50,0.00}{##1}}}
\expandafter\def\csname PY@tok@nd\endcsname{\def\PY@tc##1{\textcolor[rgb]{0.67,0.13,1.00}{##1}}}
\expandafter\def\csname PY@tok@s\endcsname{\def\PY@tc##1{\textcolor[rgb]{0.73,0.13,0.13}{##1}}}
\expandafter\def\csname PY@tok@sd\endcsname{\let\PY@it=\textit\def\PY@tc##1{\textcolor[rgb]{0.73,0.13,0.13}{##1}}}
\expandafter\def\csname PY@tok@si\endcsname{\let\PY@bf=\textbf\def\PY@tc##1{\textcolor[rgb]{0.73,0.40,0.53}{##1}}}
\expandafter\def\csname PY@tok@se\endcsname{\let\PY@bf=\textbf\def\PY@tc##1{\textcolor[rgb]{0.73,0.40,0.13}{##1}}}
\expandafter\def\csname PY@tok@sr\endcsname{\def\PY@tc##1{\textcolor[rgb]{0.73,0.40,0.53}{##1}}}
\expandafter\def\csname PY@tok@ss\endcsname{\def\PY@tc##1{\textcolor[rgb]{0.10,0.09,0.49}{##1}}}
\expandafter\def\csname PY@tok@sx\endcsname{\def\PY@tc##1{\textcolor[rgb]{0.00,0.50,0.00}{##1}}}
\expandafter\def\csname PY@tok@m\endcsname{\def\PY@tc##1{\textcolor[rgb]{0.40,0.40,0.40}{##1}}}
\expandafter\def\csname PY@tok@gh\endcsname{\let\PY@bf=\textbf\def\PY@tc##1{\textcolor[rgb]{0.00,0.00,0.50}{##1}}}
\expandafter\def\csname PY@tok@gu\endcsname{\let\PY@bf=\textbf\def\PY@tc##1{\textcolor[rgb]{0.50,0.00,0.50}{##1}}}
\expandafter\def\csname PY@tok@gd\endcsname{\def\PY@tc##1{\textcolor[rgb]{0.63,0.00,0.00}{##1}}}
\expandafter\def\csname PY@tok@gi\endcsname{\def\PY@tc##1{\textcolor[rgb]{0.00,0.63,0.00}{##1}}}
\expandafter\def\csname PY@tok@gr\endcsname{\def\PY@tc##1{\textcolor[rgb]{1.00,0.00,0.00}{##1}}}
\expandafter\def\csname PY@tok@ge\endcsname{\let\PY@it=\textit}
\expandafter\def\csname PY@tok@gs\endcsname{\let\PY@bf=\textbf}
\expandafter\def\csname PY@tok@gp\endcsname{\let\PY@bf=\textbf\def\PY@tc##1{\textcolor[rgb]{0.00,0.00,0.50}{##1}}}
\expandafter\def\csname PY@tok@go\endcsname{\def\PY@tc##1{\textcolor[rgb]{0.53,0.53,0.53}{##1}}}
\expandafter\def\csname PY@tok@gt\endcsname{\def\PY@tc##1{\textcolor[rgb]{0.00,0.27,0.87}{##1}}}
\expandafter\def\csname PY@tok@err\endcsname{\def\PY@bc##1{\setlength{\fboxsep}{0pt}\fcolorbox[rgb]{1.00,0.00,0.00}{1,1,1}{\strut ##1}}}
\expandafter\def\csname PY@tok@kc\endcsname{\let\PY@bf=\textbf\def\PY@tc##1{\textcolor[rgb]{0.00,0.50,0.00}{##1}}}
\expandafter\def\csname PY@tok@kd\endcsname{\let\PY@bf=\textbf\def\PY@tc##1{\textcolor[rgb]{0.00,0.50,0.00}{##1}}}
\expandafter\def\csname PY@tok@kn\endcsname{\let\PY@bf=\textbf\def\PY@tc##1{\textcolor[rgb]{0.00,0.50,0.00}{##1}}}
\expandafter\def\csname PY@tok@kr\endcsname{\let\PY@bf=\textbf\def\PY@tc##1{\textcolor[rgb]{0.00,0.50,0.00}{##1}}}
\expandafter\def\csname PY@tok@bp\endcsname{\def\PY@tc##1{\textcolor[rgb]{0.00,0.50,0.00}{##1}}}
\expandafter\def\csname PY@tok@fm\endcsname{\def\PY@tc##1{\textcolor[rgb]{0.00,0.00,1.00}{##1}}}
\expandafter\def\csname PY@tok@vc\endcsname{\def\PY@tc##1{\textcolor[rgb]{0.10,0.09,0.49}{##1}}}
\expandafter\def\csname PY@tok@vg\endcsname{\def\PY@tc##1{\textcolor[rgb]{0.10,0.09,0.49}{##1}}}
\expandafter\def\csname PY@tok@vi\endcsname{\def\PY@tc##1{\textcolor[rgb]{0.10,0.09,0.49}{##1}}}
\expandafter\def\csname PY@tok@vm\endcsname{\def\PY@tc##1{\textcolor[rgb]{0.10,0.09,0.49}{##1}}}
\expandafter\def\csname PY@tok@sa\endcsname{\def\PY@tc##1{\textcolor[rgb]{0.73,0.13,0.13}{##1}}}
\expandafter\def\csname PY@tok@sb\endcsname{\def\PY@tc##1{\textcolor[rgb]{0.73,0.13,0.13}{##1}}}
\expandafter\def\csname PY@tok@sc\endcsname{\def\PY@tc##1{\textcolor[rgb]{0.73,0.13,0.13}{##1}}}
\expandafter\def\csname PY@tok@dl\endcsname{\def\PY@tc##1{\textcolor[rgb]{0.73,0.13,0.13}{##1}}}
\expandafter\def\csname PY@tok@s2\endcsname{\def\PY@tc##1{\textcolor[rgb]{0.73,0.13,0.13}{##1}}}
\expandafter\def\csname PY@tok@sh\endcsname{\def\PY@tc##1{\textcolor[rgb]{0.73,0.13,0.13}{##1}}}
\expandafter\def\csname PY@tok@s1\endcsname{\def\PY@tc##1{\textcolor[rgb]{0.73,0.13,0.13}{##1}}}
\expandafter\def\csname PY@tok@mb\endcsname{\def\PY@tc##1{\textcolor[rgb]{0.40,0.40,0.40}{##1}}}
\expandafter\def\csname PY@tok@mf\endcsname{\def\PY@tc##1{\textcolor[rgb]{0.40,0.40,0.40}{##1}}}
\expandafter\def\csname PY@tok@mh\endcsname{\def\PY@tc##1{\textcolor[rgb]{0.40,0.40,0.40}{##1}}}
\expandafter\def\csname PY@tok@mi\endcsname{\def\PY@tc##1{\textcolor[rgb]{0.40,0.40,0.40}{##1}}}
\expandafter\def\csname PY@tok@il\endcsname{\def\PY@tc##1{\textcolor[rgb]{0.40,0.40,0.40}{##1}}}
\expandafter\def\csname PY@tok@mo\endcsname{\def\PY@tc##1{\textcolor[rgb]{0.40,0.40,0.40}{##1}}}
\expandafter\def\csname PY@tok@ch\endcsname{\let\PY@it=\textit\def\PY@tc##1{\textcolor[rgb]{0.25,0.50,0.50}{##1}}}
\expandafter\def\csname PY@tok@cm\endcsname{\let\PY@it=\textit\def\PY@tc##1{\textcolor[rgb]{0.25,0.50,0.50}{##1}}}
\expandafter\def\csname PY@tok@cpf\endcsname{\let\PY@it=\textit\def\PY@tc##1{\textcolor[rgb]{0.25,0.50,0.50}{##1}}}
\expandafter\def\csname PY@tok@c1\endcsname{\let\PY@it=\textit\def\PY@tc##1{\textcolor[rgb]{0.25,0.50,0.50}{##1}}}
\expandafter\def\csname PY@tok@cs\endcsname{\let\PY@it=\textit\def\PY@tc##1{\textcolor[rgb]{0.25,0.50,0.50}{##1}}}

\def\PYZbs{\char`\\}
\def\PYZus{\char`\_}
\def\PYZob{\char`\{}
\def\PYZcb{\char`\}}
\def\PYZca{\char`\^}
\def\PYZam{\char`\&}
\def\PYZlt{\char`\<}
\def\PYZgt{\char`\>}
\def\PYZsh{\char`\#}
\def\PYZpc{\char`\%}
\def\PYZdl{\char`\$}
\def\PYZhy{\char`\-}
\def\PYZsq{\char`\'}
\def\PYZdq{\char`\"}
\def\PYZti{\char`\~}
% for compatibility with earlier versions
\def\PYZat{@}
\def\PYZlb{[}
\def\PYZrb{]}
\makeatother


    % Exact colors from NB
    \definecolor{incolor}{rgb}{0.0, 0.0, 0.5}
    \definecolor{outcolor}{rgb}{0.545, 0.0, 0.0}



    
    % Prevent overflowing lines due to hard-to-break entities
    \sloppy 
    % Setup hyperref package
    \hypersetup{
      breaklinks=true,  % so long urls are correctly broken across lines
      colorlinks=true,
      urlcolor=urlcolor,
      linkcolor=linkcolor,
      citecolor=citecolor,
      }
    % Slightly bigger margins than the latex defaults
    
    \geometry{verbose,tmargin=1in,bmargin=1in,lmargin=1in,rmargin=1in}
    
    

    \begin{document}
    
    
    \maketitle
    
    

    
    \section{Assignment\_01}\label{assignment_01}

\begin{enumerate}
\def\labelenumi{\arabic{enumi}.}
\item
  用 Weka 軟體對 mushrooms.arff 利用 Naïve Bayes 進行 Supervised
  learning 選擇 ''Use training set'', 設定 Attribute: type 為 Output,
  在過程中對重要步驟截圖加以說明並回答以下問題:

  \begin{itemize}
  \item
    \begin{enumerate}
    \def\labelenumii{(\alph{enumii})}
    \tightlist
    \item
      解釋 Classifier Output, Test data 的錯誤率為多少?有多少 Test
      dataset instances 被分類到有毒 poisonous 但實際上屬於可食用的
      edible 請利用 Confusion matrix 解釋。(25\%)
    \end{enumerate}
  \item
    \begin{enumerate}
    \def\labelenumii{(\alph{enumii})}
    \setcounter{enumii}{1}
    \tightlist
    \item
      在 Output predictions 結果中 "+"
      代表的意義為何,請截圖並解釋。(10\%)
    \end{enumerate}
  \item
    \begin{enumerate}
    \def\labelenumii{(\alph{enumii})}
    \setcounter{enumii}{2}
    \tightlist
    \item
      請使用 Visualize Classifier Errors, 解釋此圖與 Confusion
      matrix之間的關係。(15\%)
    \end{enumerate}
  \end{itemize}
\end{enumerate}

    \begin{enumerate}
\def\labelenumi{\arabic{enumi}.}
\setcounter{enumi}{1}
\item
  用 python 對 mushrooms.csv 進行 Supervised learning 中的 Naïve Bayes
  分析 並回答以下問題:

  \begin{itemize}
  \item
    \begin{enumerate}
    \def\labelenumii{(\alph{enumii})}
    \tightlist
    \item
      在過程中對所有重要程式步驟進行截圖並 加以說明,越詳盡越好。 (15\%)
    \end{enumerate}
  \item
    \begin{enumerate}
    \def\labelenumii{(\alph{enumii})}
    \setcounter{enumii}{1}
    \tightlist
    \item
      請問 mushrooms 資料集中共有多少 instance?
      是否包含空值的欄位(null)? (10\%)
    \end{enumerate}
  \item
    \begin{enumerate}
    \def\labelenumii{(\alph{enumii})}
    \setcounter{enumii}{2}
    \tightlist
    \item
      請問欄位 stalk\_color\_above\_ring 有幾種不同的 value? (5\%)
    \end{enumerate}
  \item
    \begin{enumerate}
    \def\labelenumii{(\alph{enumii})}
    \setcounter{enumii}{3}
    \tightlist
    \item
      請利用 metrics.confusion\_matrix()
      呈現出混淆矩陣,並截圖加以說明。 (10\%)
    \end{enumerate}
  \item
    \begin{enumerate}
    \def\labelenumii{(\alph{enumii})}
    \setcounter{enumii}{4}
    \tightlist
    \item
      請利用 metrics.classification\_report 列出模型的準確率並與 Weka
      的結果比較何者較高? (10\%)
    \end{enumerate}
  \end{itemize}
\end{enumerate}

繳交期限 : 3/25 (三) 中午 12:00\\
請轉檔為 PDF 格式 , 檔名為 ECT\_HW1\_學號\_版本.pdf並同時附上 python 的
.ipynb 檔,命名格式同上。

    \subsection{Weka}\label{weka}

\subsubsection{(a) 錯誤率 \& Confusion
matrix}\label{a-ux932fux8aa4ux7387-confusion-matrix}

Weka的Input和參數設定如下:
\includegraphics{https://i.imgur.com/MbpV4tF.jpg}

我們使用training set來作為testing data來跑Naïve
Bayes方法,使用weka跑出來的部分result如下:

\begin{Shaded}
\begin{Highlighting}[]
\OperatorTok{===}\NormalTok{ Summary }\OperatorTok{===}

\NormalTok{Correctly Classified Instances        }\DecValTok{7984}               \FloatTok{98.2767} \OperatorTok \OperatorTok{<--}\NormalTok{ 錯誤率}
\NormalTok{Kappa statistic                          }\FloatTok{0.9655}
\NormalTok{Mean absolute error                      }\FloatTok{0.0222}
\NormalTok{Root mean squared error                  }\FloatTok{0.1209}
\NormalTok{Relative absolute error                  }\FloatTok{4.439}  \OperatorTok
\NormalTok{Total Number of Instances             }\DecValTok{8124}     

\OperatorTok{===}\NormalTok{ Confusion Matrix }\OperatorTok{===}

\NormalTok{    a    b   }\OperatorTok{<--}\NormalTok{ classified }\ImportTok{as}
 \DecValTok{3822}   \DecValTok{94} \OperatorTok{|}\NormalTok{    a }\OperatorTok{=}\NormalTok{ poisonous}
   \DecValTok{46} \DecValTok{4162} \OperatorTok{|}\NormalTok{    b }\OperatorTok{=}\NormalTok{ edible}
\end{Highlighting}
\end{Shaded}

由Incorrectly Classified Instances得知,其中錯誤率為\textbf{1.7233
\%},8124筆資料中共有140個錯誤。正確率是98.2767 \%,有7984筆正確的預測。

再透過Confusion
Matrix的結果判斷,當香菇為可食用的edible,但是被誤分類為有毒的poisonous的情況是Matrix中的左下角的區塊,數量有\textbf{46}個。

Confusion Matrix詳細的解說在第c小題會介紹。

    \subsection{(b) 在 Output predictions 結果中 "+"
代表的意義為何,請截圖並解釋。}\label{b-ux5728-output-predictions-ux7d50ux679cux4e2d-ux4ee3ux8868ux7684ux610fux7fa9ux70baux4f55ux8acbux622aux5716ux4e26ux89e3ux91cb}

勾選Output predictions為Plane text形式,部分的output結果如下:
\textgreater{}附註 1:p的意思是label是p(有毒的),轉成數值代號1。

\begin{Shaded}
\begin{Highlighting}[]
\OperatorTok{===}\NormalTok{ Predictions on training set }\OperatorTok{===}

\NormalTok{    inst#     actual  predicted error prediction}
        \DecValTok{1}        \DecValTok{1}\OperatorTok{:}\NormalTok{p        }\DecValTok{1}\OperatorTok{:}\NormalTok{p       }\FloatTok{0.985} 
        \DecValTok{2}        \DecValTok{2}\OperatorTok{:}\NormalTok{e        }\DecValTok{2}\OperatorTok{:}\NormalTok{e       }\DecValTok{1} 
        \DecValTok{3}        \DecValTok{2}\OperatorTok{:}\NormalTok{e        }\DecValTok{2}\OperatorTok{:}\NormalTok{e       }\DecValTok{1} 
        \DecValTok{4}        \DecValTok{1}\OperatorTok{:}\NormalTok{p        }\DecValTok{1}\OperatorTok{:}\NormalTok{p       }\FloatTok{0.972} 
        \DecValTok{5}        \DecValTok{2}\OperatorTok{:}\NormalTok{e        }\DecValTok{2}\OperatorTok{:}\NormalTok{e       }\DecValTok{1} 
        \DecValTok{6}        \DecValTok{2}\OperatorTok{:}\NormalTok{e        }\DecValTok{2}\OperatorTok{:}\NormalTok{e       }\DecValTok{1} 
        \DecValTok{7}        \DecValTok{2}\OperatorTok{:}\NormalTok{e        }\DecValTok{2}\OperatorTok{:}\NormalTok{e       }\DecValTok{1} 
        \DecValTok{8}        \DecValTok{2}\OperatorTok{:}\NormalTok{e        }\DecValTok{2}\OperatorTok{:}\NormalTok{e       }\DecValTok{1} 
        \DecValTok{9}        \DecValTok{1}\OperatorTok{:}\NormalTok{p        }\DecValTok{1}\OperatorTok{:}\NormalTok{p       }\FloatTok{0.973} 
       \DecValTok{10}        \DecValTok{2}\OperatorTok{:}\NormalTok{e        }\DecValTok{2}\OperatorTok{:}\NormalTok{e       }\DecValTok{1} 
\NormalTok{       ...}
     \DecValTok{4107}        \DecValTok{1}\OperatorTok{:}\NormalTok{p        }\DecValTok{2}\OperatorTok{:}\NormalTok{e   }\OperatorTok{+}   \DecValTok{1} 
\NormalTok{       ...}
     \DecValTok{4277}        \DecValTok{2}\OperatorTok{:}\NormalTok{e        }\DecValTok{1}\OperatorTok{:}\NormalTok{p   }\OperatorTok{+}   \FloatTok{0.906} 
\NormalTok{       ...}
\end{Highlighting}
\end{Shaded}

"+"號代表在Testing的過程中有error,會在error欄位顯示。

可以發現前10筆的資料都預測正確,一直到第4107筆測試資料才出現第一個Prediction
error,他的真實值是1:p,是有毒的香菇,但是Naïve
Bayes模型卻預測錯誤成2:e可食用的,屬於FN(false
negative)的錯誤。然後第4277筆測試資料出現第二個Prediction
error,他的真實值是2:e,是有可食用的香菇,但模型預測他為有毒的,屬於FP(false
positive)的錯誤。

附註:預測錯誤都是false類型。

    \subsection{(c) 請使用 Visualize Classifier Errors, 解釋此圖與 Confusion
matrix之間的關係。}\label{c-ux8acbux4f7fux7528-visualize-classifier-errors-ux89e3ux91cbux6b64ux5716ux8207-confusion-matrixux4e4bux9593ux7684ux95dcux4fc2}

Confusion
Matrix的是一個Table,表示Classerifier預測的效能。我們也可以透過圖示(visual
classifier
errors)來看到所有分布在2維平面上的預測值和真實值的關係,將虛擬的演算法用圖像化的方式呈現他的效能。

比較一下Confusion Matrix和Visualized的輸出,如果將原本的Confusion
Matrix轉換一下位置/方向,讓xy軸跟Visualized error的一樣。

\texttt{原本}

\begin{Shaded}
\begin{Highlighting}[]
\NormalTok{    p    e   }\OperatorTok{<--}\NormalTok{ classified }\ImportTok{as}
 \DecValTok{3822}   \DecValTok{94} \OperatorTok{|}\NormalTok{    p }\OperatorTok{=}\NormalTok{ poisonous}
   \DecValTok{46} \DecValTok{4162} \OperatorTok{|}\NormalTok{    e }\OperatorTok{=}\NormalTok{ edible}
   
\NormalTok{ (Predicted)}
\NormalTok{   TP   FN }\OperatorTok{|}\NormalTok{ (Actual) }
\NormalTok{   FP   TN }\OperatorTok{|}
\end{Highlighting}
\end{Shaded}

\texttt{調整後}

\begin{Shaded}
\begin{Highlighting}[]
\NormalTok{  (Actual)  }
\NormalTok{    p    e}
\NormalTok{   FP   TN }\OperatorTok{|} \AttributeTok{e}\NormalTok{ (Predicted) }
\NormalTok{   TP   FN }\OperatorTok{|}\NormalTok{ p  p }\OperatorTok{=}\NormalTok{ positive}
   \OperatorTok{--------}
   \DecValTok{46} \DecValTok{4162} \OperatorTok{|}\NormalTok{ e}
 \DecValTok{3822}   \DecValTok{94} \OperatorTok{|}\NormalTok{ p}
\end{Highlighting}
\end{Shaded}

\texttt{Visualized\ error\ 正確的預測}是用黑筆框起來的2塊區域,左下角是TP、有上角是TN,True開頭的都是正確的預測,使用叉叉來表示。而方形則是錯誤的預測。由於我們將Matrix的xy軸方向顛倒了,數值都可以直接對上Matrix的圖示。
\includegraphics{https://i.imgur.com/4sIYMaw.jpg}

    \subsection{Python}\label{python}

    由於第a小題的內容比較多,我放到最後面說明\textasciitilde{}

\subsubsection{(b) 請問 mushrooms 資料集中共有多少 instance?
是否包含空值的欄位(null)?
(10\%)}\label{b-ux8acbux554f-mushrooms-ux8cc7ux6599ux96c6ux4e2dux5171ux6709ux591aux5c11-instance-ux662fux5426ux5305ux542bux7a7aux503cux7684ux6b04ux4f4dnull-10}

這邊使用\texttt{dataFrame.info}看有幾筆資料、每個欄位的資料型別是什麼(int,
float..)、有無空值(null)的存在、佔據多少記憶體。

資料集中共有\textbf{8123}筆資料,沒有包含空值的欄位。

\begin{Shaded}
\begin{Highlighting}[]
\OperatorTok{>>>}\NormalTok{ data.info()}

\OperatorTok{<}\KeywordTok{class} \StringTok{'pandas.core.frame.DataFrame'}\OperatorTok{>}
\NormalTok{RangeIndex: }\DecValTok{8124}\NormalTok{ entries, }\DecValTok{0}\NormalTok{ to }\DecValTok{8123}              \OperatorTok{<--}\NormalTok{ 有8123筆instance}
\NormalTok{Data columns (total }\DecValTok{12}\NormalTok{ columns):                 }\OperatorTok{<--}\NormalTok{ 有12種attribute}
\BuiltInTok{type}                      \DecValTok{8124}\NormalTok{ non}\OperatorTok{-}\NormalTok{null }\BuiltInTok{object}   \OperatorTok{<--}\NormalTok{ 通通都沒有空值}
\NormalTok{cap_shape                 }\DecValTok{8124}\NormalTok{ non}\OperatorTok{-}\NormalTok{null }\BuiltInTok{object}
\NormalTok{cap_surface               }\DecValTok{8124}\NormalTok{ non}\OperatorTok{-}\NormalTok{null }\BuiltInTok{object}
\NormalTok{cap_color                 }\DecValTok{8124}\NormalTok{ non}\OperatorTok{-}\NormalTok{null }\BuiltInTok{object}
\NormalTok{odor                      }\DecValTok{8124}\NormalTok{ non}\OperatorTok{-}\NormalTok{null }\BuiltInTok{object}
\NormalTok{stalk_shape               }\DecValTok{8124}\NormalTok{ non}\OperatorTok{-}\NormalTok{null }\BuiltInTok{object}
\NormalTok{stalk_color_above_ring    }\DecValTok{8124}\NormalTok{ non}\OperatorTok{-}\NormalTok{null }\BuiltInTok{object}
\NormalTok{stalk_color_below_ring    }\DecValTok{8124}\NormalTok{ non}\OperatorTok{-}\NormalTok{null }\BuiltInTok{object}
\NormalTok{ring_number               }\DecValTok{8124}\NormalTok{ non}\OperatorTok{-}\NormalTok{null }\BuiltInTok{object}
\NormalTok{ring_type                 }\DecValTok{8124}\NormalTok{ non}\OperatorTok{-}\NormalTok{null }\BuiltInTok{object}
\NormalTok{population                }\DecValTok{8124}\NormalTok{ non}\OperatorTok{-}\NormalTok{null }\BuiltInTok{object}
\NormalTok{habitat                   }\DecValTok{8124}\NormalTok{ non}\OperatorTok{-}\NormalTok{null }\BuiltInTok{object}
\NormalTok{dtypes: }\BuiltInTok{object}\NormalTok{(}\DecValTok{12}\NormalTok{)}
\NormalTok{memory usage: }\FloatTok{761.8}\OperatorTok{+}\NormalTok{ KB}
\end{Highlighting}
\end{Shaded}

    \subsubsection{(c) 請問欄位 stalk\_color\_above\_ring 有幾種不同的
value?
(5\%)}\label{c-ux8acbux554fux6b04ux4f4d-stalk_color_above_ring-ux6709ux5e7eux7a2eux4e0dux540cux7684-value-5}

這邊使用\texttt{dataFrame.describe}看資料的平均值、分佈情況、是否有資料傾斜Skew的問題。

欄位\texttt{stalk\_color\_above\_ring}有9個unique的value,代表在此欄位8124筆資料中只有9種不同的值,代表9種不同的顏色。眾數是w(white)白色,出現4464次。
\includegraphics{https://i.imgur.com/HdHZfKg.jpg}

    \subsubsection{(d) 請利用 metrics.confusion\_matrix()
呈現出混淆矩陣,並截圖加以說明。
(10\%)}\label{d-ux8acbux5229ux7528-metrics.confusion_matrix-ux5448ux73feux51faux6df7ux6dc6ux77e9ux9663ux4e26ux622aux5716ux52a0ux4ee5ux8aaaux660e-10}

由於官方文件並沒有說明postive和negative的位置,所以我們參考維基百科來做解釋。

得知預測正確(True)的有3296+3423=6719筆資料,預測錯誤(False)的有912+493=1405筆資料。其中左上角的3296筆是True
Positive,實際有毒預測也有毒;右上角的912筆是False
Negative,實際有毒但是預測可以吃,剩下以此類推。

\begin{verbatim}
    (Predicted)
       p     n
     True False | p (Actual) 
    False  True | n
    -----------
      TP    FN  | p
      FP    TN  | n
    
   p = positive = posinious = 1
   n = negative = edible = 0
\end{verbatim}

程式碼和輸出:

\begin{Shaded}
\begin{Highlighting}[]
\OperatorTok{>>>} \BuiltInTok{print}\NormalTok{(metrics.confusion_matrix(expected, predicted))}

\NormalTok{[[}\DecValTok{3296}  \DecValTok{912}\NormalTok{]}
\NormalTok{ [ }\DecValTok{493} \DecValTok{3423}\NormalTok{]]}
\end{Highlighting}
\end{Shaded}

    \subsubsection{(e) 請利用 metrics.classification\_report
列出模型的準確率並與 Weka 的結果比較何者較高?
(10\%)}\label{e-ux8acbux5229ux7528-metrics.classification_report-ux5217ux51faux6a21ux578bux7684ux6e96ux78baux7387ux4e26ux8207-weka-ux7684ux7d50ux679cux6bd4ux8f03ux4f55ux8005ux8f03ux9ad8-10}

函數用於顯示主要分類指標的文本報告.在報告中顯示每個類的精確度,召回率,F1值等信息。我們得知python的精確度為0.83,沒有weka的0.98高。但是由於是使用Training
set來做測試weka的準確率太高反而有overfitting的問題。

\begin{Shaded}
\begin{Highlighting}[]
\OperatorTok{>>>} \ImportTok{from}\NormalTok{ sklearn }\ImportTok{import}\NormalTok{ metrics}
\OperatorTok{>>>} \BuiltInTok{print}\NormalTok{(metrics.classification_report(expected, predicted))}

\NormalTok{              precision    recall  f1}\OperatorTok{-}\NormalTok{score   support}

           \DecValTok{0}       \FloatTok{0.87}      \FloatTok{0.78}      \FloatTok{0.82}      \DecValTok{4208} \OperatorTok{<--}\NormalTok{ edible}
           \DecValTok{1}       \FloatTok{0.79}      \FloatTok{0.87}      \FloatTok{0.83}      \DecValTok{3916} \OperatorTok{<--}\NormalTok{ posinious}

\NormalTok{    accuracy                           }\FloatTok{0.83}      \DecValTok{8124} \OperatorTok{<--}\NormalTok{ 準確率}
\NormalTok{   macro avg       }\FloatTok{0.83}      \FloatTok{0.83}      \FloatTok{0.83}      \DecValTok{8124}
\NormalTok{weighted avg       }\FloatTok{0.83}      \FloatTok{0.83}      \FloatTok{0.83}      \DecValTok{8124}
\end{Highlighting}
\end{Shaded}

    \subsubsection{(a) 在過程中對所有重要程式步驟進行截圖並
加以說明,越詳盡越好。
(15\%)}\label{a-ux5728ux904eux7a0bux4e2dux5c0dux6240ux6709ux91cdux8981ux7a0bux5f0fux6b65ux9a5fux9032ux884cux622aux5716ux4e26-ux52a0ux4ee5ux8aaaux660eux8d8aux8a73ux76e1ux8d8aux597d-15}

於python檔有詳盡的註解和官方文件的連結,這裡就只講解流程了。

首先使用pandas讀入csv檔案,並切分成input和output兩個dataFrame。因為Naive
Bayes需要將nominal的資料轉成數值資料(numeric),所以要再進一步整理資料,因此使用\texttt{le.fit\_transform()}方法。我們也可以透過\texttt{le.inverse\_transform()}來看標籤對應的值是甚麼,得知0是e(edible),1是p(posinous)。


    % Add a bibliography block to the postdoc
    
    
    
    \end{document}
